
\documentclass[reqno,11pt]{amsart}

%\usepackage{color,graphicx}
%\usepackage{mathrsfs,amsbsy}
\usepackage{amssymb}
\usepackage{amsmath}
\usepackage{amsfonts}
\usepackage{bm}
\usepackage{graphicx}
\usepackage{amsthm}
\usepackage{enumerate}
\usepackage[mathscr]{eucal}
\usepackage{float}
\usepackage{mathrsfs}
\usepackage{multicol}
\usepackage{multirow}
\usepackage[all,pdf]{xy}
\usepackage[a4paper,left=3cm,right=3cm]{geometry}
\usepackage[table,xcdraw]{xcolor} % before tikz-cd
\usepackage{tikz-cd}
%\usepackage[notcite,notref]{showkeys}

% showkeys  make label explicit on the paper

\makeatletter
\@namedef{subjclassname@2010}{%
  \textup{2010} Mathematics Subject Classification}
\makeatother

\numberwithin{equation}{section}

\theoremstyle{plain}
\newtheorem{theorem}{Theorem}[section]
\newtheorem{lemma}[theorem]{Lemma}
\newtheorem{proposition}[theorem]{Proposition}
\newtheorem{corollary}[theorem]{Corollary}
\newtheorem{claim}[theorem]{Claim}
\newtheorem{defn}[theorem]{Definition}
\newtheorem{ques}[theorem]{Question}
\newtheorem*{fact}{Facts}
\newtheorem{eg}[theorem]{Example}

\theoremstyle{plain}
\newtheorem{thmsub}{Theorem}[subsection]
\newtheorem{lemmasub}[thmsub]{Lemma}
\newtheorem{corollarysub}[thmsub]{Corollary}
\newtheorem{propositionsub}[thmsub]{Proposition}
\newtheorem{defnsub}[thmsub]{Definition}

\numberwithin{equation}{section}


\theoremstyle{remark}

\newtheorem{remark}[theorem]{Remark}
\newtheorem{remarks}{Remarks}

%\renewcommand\thefootnote{\fnsymbol{footnote}}
%dont use number as footnote symbol, use this command to change

\DeclareMathOperator{\supp}{supp}
\DeclareMathOperator{\dist}{dist}
\DeclareMathOperator{\vol}{vol}
\DeclareMathOperator{\diag}{diag}
\DeclareMathOperator{\tr}{tr}
\DeclareMathOperator{\Img}{\operatorname{Im}}
\DeclareMathOperator{\Id}{\operatorname{Id}}
\DeclareMathOperator{\Rep}{\operatorname{Rep}}
\DeclareMathOperator{\Mod}{\operatorname{Mod}}
\DeclareMathOperator{\Hom}{\operatorname{Hom}}
\DeclareMathOperator{\Ext}{\operatorname{Ext}}
\DeclareMathOperator{\gldim}{\operatorname{gl.dim}}
\DeclareMathOperator{\projdim}{\operatorname{proj.dim}}
\DeclareMathOperator{\injdim}{\operatorname{inj.dim}}
\DeclareMathOperator{\dimv}{\operatorname{\underline{\mathbf{dim}}}}

\newcommand{\Spec}{\operatorname{Spec}}
\DeclareMathOperator{\Flagd}{\operatorname{Flag}_{\mathbf{d}}}
\DeclareMathOperator{\Flagdstr}{\operatorname{Flag}_{\mathbf{d},str}}
\newcommand{\Gr}{\operatorname{Gr}}
\newcommand{\Grr}{\operatorname{Gr}}
\newcommand{\Grq}{\operatorname{Gr}^{KQ}}
\newcommand{\Flag}[1]{\operatorname{Flag}_{\mathbf{#1}}}
\newcommand{\Flagstr}[1]{\operatorname{Flag}_{\mathbf{#1},str}}
\newcommand{\dimvec}[1]{\mathbf{#1}}
\newcommand{\ord}{\operatorname{ord}}
\newcommand{\orde}{\operatorname{ord}_e }
\newcommand{\representation}[2]{\genfrac{}{}{0pt}{3}{\phantom{000}#2\phantom{00}}{#1}}

\setlength\intextsep{0cm}
\setlength\textfloatsep{0cm}
\begin{document}
\date{}

\title
{Talk 4: Admissible canonical bundle I
}


\author{Xiaoxiang Zhou}
\address{School of Mathematical Sciences\\
University of Bonn\\
Bonn, 53115\\ Germany\\} 
\email{email:xx352229@mail.ustc.edu.cn}




\maketitle
\tableofcontents
%%%%%%%%%%%%%%%%%%%%%%%%%%%%%%%%%%%%%%%%%%%%%%%%%%%%%%%%%%%%%%%%%%%%%%%%%%%%%%%%%%%%%%%%%%%%%

First of all, I should apologize that I didn't prepare so well for this talk(so I also \LaTeX$\,$ them after the talk). Everythime when I was reading the paper \cite{yuan2021arithmetic}, I felt myself so foolish. Feel free to use the black boxes in the talk. 

This note records contents in the talk, so it may contain something showed by Prof. Peter Scholze, not by me. The initial goal of this note is to keep them somewhere, so that some day when the listeners or I want to recapture something in the talk, we can easily get them from this note. Feel free to ask me questions and give me hundreds of typos!

\section{Berkovich space}

I copied some words from \cite[3.1.1]{yuan2021arithmetic} in the talk to give an extremely short introduction to the Berkovich space. 

Idea: the point of a scheme can be viewed as the map to the field, so is the point of the Berkovich space(the special seminorm). If you have not heard about the Berkovich space, you can replace $X^{an}$ to $X(K)$ in the following sections.

\section{metric and measure}
\subsection{metric of line bundle}
\subsection{the Chambert-Loir measure}
\section{statement of the main theorem}
\section{uniqueness}
\section{existence--unfinished}
\bibliographystyle{plain}
\bibliography{reference}
\end{document}




