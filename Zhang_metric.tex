
\documentclass[reqno,11pt]{amsart}

%\usepackage{color,graphicx}
%\usepackage{mathrsfs,amsbsy}
\usepackage{amssymb}
\usepackage{amsmath}
\usepackage{amsfonts}
\usepackage{bm}
\usepackage{graphicx}
\usepackage{amsthm}
\usepackage{enumerate}
\usepackage[mathscr]{eucal}
\usepackage{float}
\usepackage{mathrsfs}
\usepackage{multicol}
\usepackage{multirow}
\usepackage[all,pdf]{xy}
\usepackage[a4paper,left=3cm,right=3cm]{geometry}
\usepackage[table,xcdraw]{xcolor} % before tikz-cd
\usepackage{tikz-cd}
\usepackage{hyperref}
%\usepackage[notcite,notref]{showkeys}

% showkeys  make label explicit on the paper

\makeatletter
\@namedef{subjclassname@2010}{%
  \textup{2010} Mathematics Subject Classification}
\makeatother

\numberwithin{equation}{section}

\theoremstyle{plain}
\newtheorem{theorem}{Theorem}[section]
\newtheorem{lemma}[theorem]{Lemma}
\newtheorem{proposition}[theorem]{Proposition}
\newtheorem{corollary}[theorem]{Corollary}
\newtheorem{claim}[theorem]{Claim}
\newtheorem{defn}[theorem]{Definition}
\newtheorem{ques}[theorem]{Question}
\newtheorem*{bbox}{Black box}
\newtheorem{eg}[theorem]{Example}

\theoremstyle{plain}
\newtheorem{thmsub}{Theorem}[subsection]
\newtheorem{lemmasub}[thmsub]{Lemma}
\newtheorem{corollarysub}[thmsub]{Corollary}
\newtheorem{propositionsub}[thmsub]{Proposition}
\newtheorem{defnsub}[thmsub]{Definition}

\numberwithin{equation}{section}


\theoremstyle{remark}

\newtheorem{remark}[theorem]{Remark}
\newtheorem{remarks}{Remarks}

%\renewcommand\thefootnote{\fnsymbol{footnote}}
%dont use number as footnote symbol, use this command to change

\DeclareMathOperator{\supp}{supp}
\DeclareMathOperator{\dist}{dist}
\DeclareMathOperator{\vol}{vol}
\DeclareMathOperator{\diag}{diag}
\DeclareMathOperator{\tr}{tr}
\DeclareMathOperator{\Img}{\operatorname{Im}}
\DeclareMathOperator{\Id}{\operatorname{Id}}
\DeclareMathOperator{\Rep}{\operatorname{Rep}}
\DeclareMathOperator{\Mod}{\operatorname{Mod}}
\DeclareMathOperator{\Hom}{\operatorname{Hom}}
\DeclareMathOperator{\Ext}{\operatorname{Ext}}
\DeclareMathOperator{\gldim}{\operatorname{gl.dim}}
\DeclareMathOperator{\projdim}{\operatorname{proj.dim}}
\DeclareMathOperator{\injdim}{\operatorname{inj.dim}}
\DeclareMathOperator{\dimv}{\operatorname{\underline{\mathbf{dim}}}}
\DeclareMathOperator{\Pic}{\operatorname{Pic}}
\DeclareMathOperator{\Jac}{\operatorname{Jac}}
\newcommand{\Spec}{\operatorname{Spec}}
\DeclareMathOperator{\Flagd}{\operatorname{Flag}_{\mathbf{d}}}
\DeclareMathOperator{\Flagdstr}{\operatorname{Flag}_{\mathbf{d},str}}
\newcommand{\Gr}{\operatorname{Gr}}
\newcommand{\Grr}{\operatorname{Gr}}
\newcommand{\Grq}{\operatorname{Gr}^{KQ}}
\newcommand{\Flag}[1]{\operatorname{Flag}_{\mathbf{#1}}}
\newcommand{\Flagstr}[1]{\operatorname{Flag}_{\mathbf{#1},str}}
\newcommand{\dimvec}[1]{\mathbf{#1}}
\newcommand{\norm}[1]{\Vert{#1}\Vert}
\newcommand{\ord}{\operatorname{ord}}
\newcommand{\orde}{\operatorname{ord}_e }
\newcommand{\representation}[2]{\genfrac{}{}{0pt}{3}{\phantom{000}#2\phantom{00}}{#1}}

\setlength\intextsep{0cm}
\setlength\textfloatsep{0cm}
\begin{document}
\date{}

\title
{Talk 4: Admissible canonical bundle I
}


\author{Xiaoxiang Zhou}
\address{School of Mathematical Sciences\\
University of Bonn\\
Bonn, 53115\\ Germany\\} 
\email{email:xx352229@mail.ustc.edu.cn}



\setcounter{tocdepth}{1}
\maketitle
\tableofcontents
%%%%%%%%%%%%%%%%%%%%%%%%%%%%%%%%%%%%%%%%%%%%%%%%%%%%%%%%%%%%%%%%%%%%%%%%%%%%%%%%%%%%%%%%%%%%%

First of all, I should apologize that I didn't prepare so well for this talk(so I also \LaTeX$\,$ them after the talk). Everythime when I was reading the paper \cite{yuan2021arithmetic}, I felt myself so foolish. So feel free to use the black boxes in the talk, and get the general feeling before checking the details.

This note records contents in the talk, so it may contain something showed by Prof. Peter Scholze, not by me. The initial goal of this note is to keep them somewhere, so that some day when the listeners or I want to recapture something in the talk, we can easily get them from this note. Feel free to ask me questions and give me hundreds of typos!

\section{Berkovich space}

I copied some words from \cite[3.1.1]{yuan2021arithmetic} in the talk to give an extremely short introduction to the Berkovich space. 

Idea: the point of a scheme can be viewed as the map to the field, so is the point of the Berkovich space(the special seminorm). If you have not heard about the Berkovich space, you can replace $X^{an}$ to $X(K)$ in the following sections.

\section{metric and measure}
For the statement of the main theorem, one should introduce concepts of metric (over line bundle) and measure (over Berkovich space).

\subsection{metric of line bundle}
Recall that when we learn Riemann metric, we first define it fiberwise, then localwise, and finally globalwise. In today's talk we also define it similarly.

\begin{defn}[fiber of line bundle over Berkovich space]
Fix the scheme $X/K$ and the line bundle $L$ over $X$. the fiber of $L$ at point $x \in X^{an}$ is defined as
$$L^{an}(x):= H_x \otimes_{\kappa(\bar{x}) }L(\bar{x}).$$

\end{defn}
\begin{center}
% https://tikzcd.yichuanshen.de/#N4Igdg9gJgpgziAXAbVABwnAlgFyxMJZABgBoAmAXVJADcBDAGwFcYkQAPEAX1PU1z5CKchWp0mrdgB1pAI3oAnYB248+IDNjwEiARlIBmcQxZtEIABoA9YPTBre-bUKKGjJyeavrng3Sjuep5m7AAyvpoCOsLIosQhUhZhABSyCsqqAJSRWv6xZAk0pkkgYbb23CkcOdziMFAA5vBEoABmihAAtkhkIDgQSAb99FiM7F30aHADkR3dQzSziO4gjFhg3pCbIDQAFjD0UOzbbEuj4xanu2v0cjCMAArRrhYb2LA3Jd6yG3OdPUQfWWogkoQsv0INEYdwezxcARA7ywn2hGy2BDOIAORxOmJuOAueJ2ThA80BoOWq2+MmkAGspmh6P8FogAGxLQZA-aHY5XfGk8lIACsnMW2N5xLYgoBSAALGKVjzcfySZRuEA
\begin{tikzcd}[row sep={10mm,between origins}, column sep=3mm, labl/.style={rotate=-30,pos=0.65}, labl2/.style={rotate=-30,pos=0.45}]
L^{an}(x) \arrow[dd, no head]   &            & L(\bar{x}) \arrow[dd, no head]     &      \\
            &   &                   & L \arrow[dd, no head] \\[-5mm]
\hspace{-1.8cm}Spec H_x=x \arrow[rr, maps to] \arrow[rd, "\in" labl, phantom ] &            & \bar{x} \arrow[rd, "\in" labl2, phantom] &      \\
               & X^{an} \arrow[rr, "\kappa"] &  & X   
\end{tikzcd}

\end{center}
\begin{defn}[metric pointwise]
We define the metric of line bundle $L$ at $x \in X^{an}$ as the norm of $H_x$-modules $L^{an}(x)$:
$$\norm{\cdot}(x):L^{an}(x) \longrightarrow \mathbb{R}_{>0} \qquad \text{such that }\norm{fs}(x)=|f|_{H_x} \cdot\norm{s}(x) $$
%\text{ for any $f \in H_x$, $s \in L^{an}(x)$}
\end{defn}
\begin{defn}[metric globalwise]
The metric of line bundle $L$ is defined as the collections of metrics (at every point) such that the metrics vary in a continuous way. Equivalently,
$$\norm{\cdot}=\left\{ \norm{\cdot}(x)\, \middle|\, x\in X^{an} \text{ + cont} \right\}$$
where the continuity means for any open subset $U$ of $X$, any section $s \in \Gamma(U)$, the norm map
$$\norm{s}: U^{an} \longrightarrow \mathbb{R}>0 \qquad x \longmapsto \norm{s}(x):=\norm{s_x}(x)$$
is continuous.

When we have one metric $\norm{\cdot}$ over $L$, we would call the pair $(L,\norm{\cdot})$ the metrized line bundle.
\end{defn}

\begin{proposition}
Here are some easy properties of the metrized line bundle:
\begin{enumerate}[(1)]
\item The metrized line bundles over $X$ form a group by
$$(L_1,\norm{\cdot}_1) \otimes (L_2,\norm{\cdot}_2):=(L_1 \otimes L_2, \norm{\cdot}_1\cdot \norm{\cdot}_2)$$
the inverse and the idendity is also easily defined;
\item If one has the map $f:X \longrightarrow Y$ and the metrized line bundle $(L,\norm{\cdot})$ over $Y$, then we can always pull back $f$ to get a metrized line bundle $f^*(L,\norm{\cdot})$ over $X$;
\item The metric $\norm{\cdot}$ over structure sheaf $\mathcal{O}_X$ can be viewed as a positive real function on $X^{an}$.\footnote{Notice that we have the constant section $1$ of $\mathcal{O}_X$.}
\end{enumerate} 
\end{proposition}
\begin{remark}
By this definition it's not easy to define integrable metric (or semipositive metric). By translating the metrized line bundles as the adelic line bundles, we can define integrable(it's already done in Talk 3).
\end{remark}
Here the Prof. Peter Scholze explains how to translate the metrized line bundles as the adelic line bundles:


\subsection{Chambert-Loir measure}
We may not have any time to define this measure, since defining it needs to introduce quite a lot of concepts, such as arithmetic intersection degree and the arithmetic Chow group. But we can still see the intuition and view some properties as the black box.

\begin{defn}[Chambert-Loir measure, in complex case]
Suppose $C$ is a projective smooth curve over $\mathbb{C}$, and fix the metrized line bundle $(L,\norm{\cdot})$, the Chambert-Loir measure $c_1(L,\norm{\cdot})$ is defined as the $(1,1)$-form
$$c_1(L,\norm{\cdot}):= \frac{i}{2\pi}\partial\bar{\partial} \log \norm{s}^2$$
where $s$ is a local section of $L$ which is nowhere zero.  This definition doesn't depend on the choise of the section $s$.
\end{defn}
Notice that on curve $C/\mathbb{C}$, the $(1,1)$-form can be viewed as the measure. Of course we can no longer use this definition when the curve $C$ is over some non-Archimedean field, so I prepare the black box for you:
\begin{bbox}\
\begin{enumerate}[(1)]
\item $c_1(L,\norm{\cdot})$ is an measure on $C^{an}$.
\item The Chambert-Loir measure is compatible with the group structure of line bundle, i.e.
\begin{equation*}
\begin{aligned}
  &c_1(L_1\otimes L_2,\norm{\cdot}_1\cdot \norm{\cdot}_2)=c_1(L_1,\norm{\cdot}_1)+c_1(L_2,\norm{\cdot}_2),  \\
   &c_1(L^{\vee},\norm{\cdot})=-c_1(L,\norm{\cdot}^{-1}),\quad\quad c_1(\mathcal{O}_C,\norm{\cdot}_{const})=0.
\end{aligned}
\end{equation*}
\item (non-Archimedean Calabi theorem) Let $\norm{\cdot}$ be an integrable metric on $\mathcal{O}_C$, then
$$c_1(\mathcal{O}_C,\norm{\cdot})=0 \iff \norm{\cdot} \equiv C_0 $$
\item (normalization) let $(L,\norm{\cdot})$ be a metrized line bundle of degree $d$, then
$$\int_{C^{an}} c_1(L,\norm{\cdot})=d.$$
\end{enumerate}
\end{bbox}
\section{statement of the main theorem}
Fix $K$ be an non-Archimedean field,\footnote{We can suppose $K=\bar{K}$ to avoid some technical conditions.} and $C/K$ be a smooth projective curve, and $x\in C(\bar{K})$ be a point. We have the following canonical maps

\begin{figure}[th]
	\begin{minipage}[t]{.48\textwidth}
		\centering
		\begin{tikzcd}
		C \arrow[rd, "{(\Id,x)}"] &              & C \\
		C \arrow[r, "\Delta"]     & C\times C \arrow[ru, "p_1"] \arrow[rd, "p_2"] &   \\
		C \arrow[ru, "{(x,\Id)}"'] &              & C
		\end{tikzcd}
		\label{fig1}
	\end{minipage}
	\begin{minipage}[t]{.48\textwidth}
		\centering
% https://tikzcd.yichuanshen.de/#N4Igdg9gJgpgziAXAbVABwnAlgFyxMJZARgBpiBdUkANwEMAbAVxiRAGEQBfU9TXfIRQAmclVqMWbdgB0ZeALbwABJx59seAkTIAGcfWatEIORCUBzOgH12APWByaMGF268QGTYKKj91QykTOQU6HAALAGNGYAB5Lmtge2EuAAo5ABEYBhw6AEp3DQFtFF1SYQNJY1MZUIjohjiE9lSADwL1T34tIWQygGZKo2lucRgoC3giUAAzACdzJDKQHAgkMglh4JksnLpCkHnFxFEVtcRl8Jg6KDZIMFZOo4UkfupV9eorm7uCR49nkgAKzvc4bQLVNqkOQASSgHQBCxeiAALKDgV9rrcTPdHhQuEA
\begin{tikzcd}[row sep=5mm]
  &[-12mm] \omega_C^{\vee} \arrow[d, no head] & \mathcal{O}_{C^2}(\Delta) \arrow[d, no head] \\
  & C \arrow[r, "\Delta"]              & C\times C  \\[-5mm]
\mathcal{O}_C(x) \arrow[d, no head] &  &            \\
C \arrow[rruu, "{(x,\Id)}"']         &  &           
\end{tikzcd}
		\label{fig2}
	\end{minipage}
\end{figure}
and also some isomorphisms of line bundles:
\begin{equation*}
\begin{aligned}
  \Delta^* \mathcal{O}_{C^2}(\Delta)&\cong \omega_C^{\vee}  \\
  (x,\Id)^* \mathcal{O}_{C^2}(\Delta)&\cong \mathcal{O}_C(x)
\end{aligned}
\end{equation*}
If we have the metric $\norm{\cdot}_{\Delta}$ over line bundle $\mathcal{O}_{C^2}(\Delta)$, then it induces the metric $\norm{\cdot}_{\omega_C}$ and $\norm{\cdot}_{\mathcal{O}_C(x)}$ by the following ways:
\begin{equation*}
\begin{aligned}
 (\omega_C^{\vee},\norm{\cdot}_{\omega_C}^{-1}) &:= \Delta^* (\mathcal{O}_{C^2}(\Delta), \norm{\cdot}_{\Delta}) \\ 
 (\mathcal{O}_C(x),\norm{\cdot}_{\mathcal{O}_C(x)}) &:= (x,\Id)^* (\mathcal{O}_{C^2}(\Delta), \norm{\cdot}_{\Delta}) \\ 
\end{aligned}
\end{equation*}
we can also define the Green's functions as follows:
\begin{equation*}
\begin{aligned}
 &g_{\Delta}:(C \times C)^{an} \setminus \Delta^{an} \longrightarrow \mathbb{R} \qquad  &&g_{\Delta}(x,y):=-\log \big(\norm{1}_{\Delta}(x,y)\big)  \\
 &g_{x}:C^{an} \setminus {x} \longrightarrow \mathbb{R} \qquad  &&g_{x}(y):=-\log \big(\norm{1}_{x}(y)\big) =g_{\Delta}(x,y) \\
\end{aligned}
\end{equation*}
\begin{defn}[symmetric metric]
The metric $\norm{\cdot}_{\Delta}$ over line bundle $\mathcal{O}_{C^2}(\Delta)$ is symmetric if the corresponding Green's function $g_{\Delta}(x,y)$ is symmetric.
\end{defn}
In the following pages, when the norm  $\norm{\cdot}_{\Delta}$ is special, we would replace the notations $\norm{\cdot}_{\Delta}$ by $\norm{\cdot}_{\Delta,a}$, $\norm{\cdot}_{\omega_C}$ by $\norm{\cdot}_{a}$, $\norm{\cdot}_{\mathcal{O}_C(x)}$ by $\norm{\cdot}_{x}$ and $g_{\Delta}$ by $g_{\Delta,a}$. \footnote{Here the letter $a$ means "admissible". We will define the "admissible metric" over curve on Section\ref{sec:existence}, and actually "admissible metric" over $\mathcal{O}_{C^2}(\Delta)$ is defined in \cite[???]{yuan2021arithmetic}.}
\\
\\

Finally we can state our central theorem:
\begin{theorem}[Zhang's metric]\label{thm:main}
Suppose $K=\bar{K}$.\footnote{It's not an essential condition, we just don't want to worry about finite field extension.}There is a unique symmetric integrable metric $\norm{\cdot}_{\Delta,a}$ over $\mathcal{O}_{C^2}(\Delta)$ such that for any $x,y \in C(K)$, we have
\begin{enumerate}[(1)]
\item (compatibility, or admissible) We have the identities
\begin{equation*}
\begin{aligned}
  c_1(\mathcal{O}(x),\norm{\cdot}_x)&=c_1(\mathcal{O}(y),\norm{\cdot}_y)  \\ 
  (2g-2)c_1(\mathcal{O}(x),\norm{\cdot}_x)&=c_1(\omega_C,\norm{\cdot}_a)  \\ 
\end{aligned}
\end{equation*}
as the Chambert-Loir measure;
\item (normalization) We always have the equality
$$\int_{(C)^{an}} g_x(-) c_1(\mathcal{O}(x),\norm{\cdot}_x) =0.$$
\end{enumerate}
We will call this metric $\norm{\cdot}_{\Delta,a}$ and its induced metric $\norm{\cdot}_{a}$ as the Zhang's metric.
\end{theorem}

\section{uniqueness}
In this section we want to prove the uniqueness part. If we have two metrics $\norm{\cdot}_{\Delta,a}$ and $\norm{\cdot}_{\Delta,a}'$ satisfying the Theorem \ref{thm:main}, we just denote $\norm{\cdot}_a'$, $\norm{\cdot}_x'$, $g_{\Delta,a}'(x,y)$ and $g_x'(y)$ as the induced metrics or Green's functions induced by $\norm{\cdot}_{\Delta,a}'$. We also fix $x_0 \in C(K)$, and denote 
$$\frac{\norm{\cdot}_{x_0}'(y)}{\norm{\cdot}_{x_0}(y)}=e^{\varphi(y)}$$
where $\varphi:C^{an} \longrightarrow \mathbb{R}$ is a continous function on $C^{an}$.

We divide the process into four steps:
\subsection*{\underline{Step1}}
Compute $\displaystyle\frac{\norm{\cdot}_x'}{\norm{\cdot}_x}$ and $\displaystyle \frac{\norm{\cdot}_a'}{\norm{\cdot}_a}$.

We know that 
\begin{equation*}
\begin{aligned}
&c_1\left(\mathcal{O}_C,\frac{\norm{\cdot}_a'}{\norm{\cdot}_a} \cdot \left(\frac{\norm{\cdot}_x'}{\norm{\cdot}_x} \right)^{-(2g-2)}\right)\\
  =\;&c_1\Big(\omega_C,\norm{\cdot}_a'\Big)-c_1\Big(\omega_C,\norm{\cdot}_a\Big)-(2g-2)c_1\Big(\mathcal{O}(x_0),\norm{\cdot}_{x_0}'\Big)+ (2g-2)c_1\Big(\mathcal{O}(x_0),\norm{\cdot}_{x_0}\Big) \\
    =\;& 0  \\ 
\end{aligned}
\end{equation*}
By the non-Archimedean Calabi theorem, we get $\displaystyle \frac{\norm{\cdot}_a'}{\norm{\cdot}_a} \cdot \left(\frac{\norm{\cdot}_x'}{\norm{\cdot}_x} \right)^{-(2g-2)}$ is a constant, we get
$$\frac{\norm{\cdot}_a'}{\norm{\cdot}_a}=e^{(2g-2)\varphi+c_1}$$
where $c_1$ is a constant number. Similarly, one can write
$$\frac{\norm{\cdot}_x'}{\norm{\cdot}_x}=e^{\varphi+\psi(x)}$$
where $\psi:C^{an} \longrightarrow \mathbb{R}$ is a "constant number depending on $x$".
\subsection*{\underline{Step2}}
By the symmetric property, we get $\psi-\varphi\equiv c_2$ is an constant function. We compute
$$
  g_{\Delta,a}'(x,y)-g_{\Delta,a}(x,y)= g_x'(y)-g_x(y)   =-\log \frac{\norm{1}_x(y)}{\norm{1}_x'(y)}=-\psi(x)-\phi(y)
$$
thus
$$-\psi(x)-\varphi(y)=-\psi(y)-\varphi(x) \quad\Longrightarrow\quad \psi(x)-\varphi(x)=\psi(y)-\varphi(y)$$
\subsection*{\underline{Step3}}
We show that $\varphi$ is also a constant. By the following commutative diagram, we can pullback metrized line bundle $(\mathcal{O}_{C^2}(\Delta), \norm{\cdot}_{\Delta})$ in two ways, and should get the same metrized line bundle.
\begin{center}
% https://tikzcd.yichuanshen.de/#N4Igdg9gJgpgziAXAbVABwnAlgFyxMJZAJgBoBGAXVJADcBDAGwFcYkQBhEAX1PU1z5CKAMwVqdJq3YcAOrLwBbeAAIuvftjwEiZAAwSGLNohAAKeRGUBzegH0OAPWDzaMGL3mQATopeyAYygIHG47emcAWnJuAEoePhAMLSEiMQMaI2lTC1lFehwACwCmYAB5MOAnYm5cgBEYRhx6WNIvCF9-IJDK+Qam+lJ6OITNQR0UclJiQykTc3l8opLGcrCOMwAPVvbO+W7Qu23RpIFtYWQpkVnjGRPk8Yu9UmvMufZNngkYKGt4IlAADNvFYkM8QDgIEgppJbqY+o1midgaDEGQIVDEODCjB6FB2JAwGwNCAUYokGIMdCaDi8QSCMTEmSkABWGiQ6mw7LmTZtWQASSg8RJzMQABZ2Zi2SBafjTITGUCQeTEAA2SWst5wkCfEXKpDqqlYrXc3WUbhAA
\begin{tikzcd}[row sep={12mm,between origins}, column sep={30mm,between origins}]
&&[-20mm] {(\omega_C^{\vee},\norm{\cdot}_a^{-1})} \arrow[d, no head] & {(\mathcal{O}_{C^2}(\Delta),\norm{\cdot}_{\Delta,a})} \arrow[d, no head] \\
  && C \arrow[r, "\Delta"]     & C\times C        \\[-5mm]
  &|[fill=white]| {(\mathcal{O}_C(x),\norm{\cdot}_x)} \arrow[d, no head] &    &                  \\
x \arrow[r, "x"] \arrow[rruu, "x"{pos=0.3},shorten <=32mm]\arrow[rruu,no head, shorten >=19mm] & C \arrow[rruu, "{(x,\Id)}"',end anchor={[xshift=-1ex]south}] &    &                 
\end{tikzcd}
\end{center}
We get
\begin{equation*}
\begin{aligned}
& \norm{\Delta^* s}_a^{-1}=\norm{(x,\Id)^* s}_x \qquad \text{ where $s$ is the section of $\mathcal{O}_{C^2}(\Delta)$ in general place}\\
  \Longrightarrow\;& \frac{\norm{\cdot}_a'(x)}{\norm{\cdot}_a(x)}\cdot \frac{\norm{\cdot}_x'(x)}{\norm{\cdot}_x(x)} \equiv 1\\ 
  \Longrightarrow\;&
  e^{(2g-2)\varphi(x)+c_1} \cdot e^{2\varphi(x)+c_2} \equiv 1
\end{aligned}
\end{equation*}
So $\varphi$ is a constant.
\subsection*{\underline{Step4}}
Suppose $\norm{\cdot}_x'=c_3\norm{\cdot}_x$, then $c_3=1$.

We use the normalization as follows:
\begin{equation*}
\begin{aligned}
0  =\;& \int_{C^{an}}-\log \norm{\cdot}_x' \cdot c_1(\mathcal{O}(x),\norm{\cdot}_x')\\ 
  =\;& \int_{C^{an}}-\log \norm{\cdot}_x \cdot c_1(\mathcal{O}(x),\norm{\cdot}_x) -\log c_3 \\ 
  =\;&  -\log c_3 \\ 
\end{aligned}
\end{equation*}
so $\norm{\cdot}_x'=\norm{\cdot}_x$, we get finally $\norm{\cdot}_{\Delta,a}'=\norm{\cdot}_{\Delta,a}$.

\section{admissible metric over $C^{an}$}\label{sec:existence}

For the existence part of the main theorem, one should define the "admissibility" of the metric. We will see that the admissable metric exists and is unique up to the multiplication of an constant. Thus the proof can be divided into two parts:
\begin{itemize}
\item find the admissible metric;
\item make the normalization.
\end{itemize}
Still, it's not easy to define and construct the admissable metric over the curve. Actually, we first define the admissible metric over the Abelian variety, and then embed the curve into its Jacobian to get the "admissibility" property.

\begin{defn}[admissible metric over Abelian variety]
Let $(L,\norm{\cdot}_L)$ be the metrized line bundle over the Abelian variety $A$. We define the admissibility step by step:
\begin{itemize}
\item  when $L$ is even, i.e. $[-1]^*L\cong L$, we get $[2]^*L\cong L^{\otimes 4}$. the norm $\norm{\cdot}_L$ is called admissible when this isomorphism induces the isomorphism of metrized line bundle, i.e.
$$[2]^*(L,\norm{\cdot}_L)\cong (L,\norm{\cdot}_L)^{\otimes 4};$$ 
\item  when $L$ is odd, i.e. $[-1]^*L\cong L^{\vee}$, we get $[2]^*L\cong L^{\otimes 2}$. the norm $\norm{\cdot}_L$ is called admissible when this isomorphism induces the isomorphism of metrized line bundle, i.e.
$$[2]^*(L,\norm{\cdot}_L)\cong (L,\norm{\cdot}_L)^{\otimes 2};$$ 
\item In general, $\norm{\cdot}_L$ is admissible if and only if the induced metric on $L \otimes [-1]^*L$ and $L \otimes ([-1]^*L)^{\vee}$ are both admissible.
\end{itemize}
\end{defn}
\begin{remark}
Fix the line bundle $L$ over $A$, the admissible metric over $L$ exists and is unique(up to a constant) by the Tate's limiting argument. In some sense, you take the average of the metric again and again, and finally you'll get the "most averaged" metric, which is now called the "admissible metric".
\end{remark}
\begin{proposition}
If the metrized line bundle $(L,\norm{\cdot})$ over $A$ is admissible, then $(L,\norm{\cdot})$ is integrable.
\end{proposition}
\begin{remark}
The admissibility is also compatible with the group stucture of Picard group.
\end{remark}
Now we can define the admissibility over curve.
\begin{defn}[admissible metric over curve]
Fix $\alpha \in \Pic^1(C)$, it induce an embedding $i_{\alpha}:C \hookrightarrow J:=\Jac(C)$ and also a theta divisor $\theta_{\alpha}$ over $J$.\footnote{Remember that our curve is of genus $g>0$.} 

The metrized line bundle $(L,\norm{\cdot}_L)$ over the curve $C^{an}$ is called admissible if there exists an metrized line bundle $(M,\norm{\cdot}_M)$ over $J$, such that 
\begin{itemize}
\item $(M,\norm{\cdot}_M)$ is admissible;
\item $M$ is algebraic equivalent to $\mathcal{O}_J(n\theta_{\alpha})$;
\item There exists $m \in \mathbb{Z}_{>0}$, such that $$i_{\alpha}^*(M,\norm{\cdot}_M) \cong (L,\norm{\cdot}_L)^{\otimes m}.$$
\end{itemize}
\end{defn}
\begin{center}
% https://tikzcd.yichuanshen.de/#N4Igdg9gJgpgziAXAbVABwnAlgFyxMJZABgBoBGAXVJADcBDAGwFcYkQAZEAX1PU1z5CKcqQBM1Ok1bsAwjz4gM2PASJiKkhizaIQAWQX8VQogGZxW6bpAApI0oGrhyACyaa2mXoA6PgLb0OAAWAMZMwADy3AD6tgAUYH4hMDj0McB+TGjB9NwAlA7KgmooGsRWOux+kABO-pk+oVAQOLGGvMYlLmQVntbVPnUNfs2tsVzckjBQAObwRKAAZrUQ-khkIDgQSKIgwTD0UOyQYGydICtruzTbSBZSVXpYGVmMOXkOV+uIGls7iAejHoACMYIwAApOUx6RgwJY4EA0A5HE4Ec6Kb5Idz-e7Iw7HPSnDHLVY-ABstwBmxRhPA6K+ZKQAFYqUg-rS0WceJRuEA
\begin{tikzcd}[column sep=0mm, row sep=3mm]
\norm{\cdot}_L \arrow[d, no head] & [-3mm]& \norm{\cdot}_M \arrow[d, no head] &[-3mm]   &                \\
L \arrow[rd, no head]             & & M \arrow[rr, draw=none, "\sim"{description}]\arrow[rd, no head]             &   & \mathcal{O}_J(n\theta_{\alpha}) \arrow[ld, no head] \\
& C \arrow[rr, "i_{\alpha}"] &     & J&               
\end{tikzcd}
\end{center}
\begin{remark}
Fix the line bundle $L$ over $C$, the admissible metric over $L$ exists and is unique(up to a constant). For the uniqueness we can refer to the non-Archimedean Calabi theorem, but I'm still not sure how to prove the existence. Maybe you'll find the answer \href{http://qirui.li/main.pdf}{here}.
\end{remark}
\begin{remark}
The definition of admissible is well-defined, i.e. doesn't depend on the choice of $\alpha \in \Pic^1(C)$. It's stable under the base change(when we assume $K=\bar{K}$ I guess we don't need this property) and compatible with the group structure in $\Pic(C)$.
\end{remark}
The following proposition gives the key property of admissible metric over curve $C$. In some sense, this property describe the admissible metric in intrinsic way.
\begin{proposition}
There exists an unique probability measure $d\mu_{\alpha}$ over $C^{an}$ such that for any metrized line bundle $(L,\norm{\cdot}_L)$ over $C^{an}$,
$$\norm{\cdot}_L \text{ is admissible } \iff \begin{aligned}
&\hspace{0.3cm}\norm{\cdot}_L \text{ is integrable}\\
&\hspace{1.5cm}+\\
& c_1(L,\norm{\cdot}_L)= \deg (L) d\mu_{\alpha}
\end{aligned}$$
\end{proposition}
The next lemma give the explicit construction of metrized line bundle $(\mathcal{O}_{C^2}(\Delta),\norm{\cdot}_{\Delta,a})'$, and we will see that this is nearly the desired metrized line bundle. For this, we denote the Poincaré line bundle $P$ on $J$ by
$$P:=-m^*\mathcal{O}_{J}(\theta_{\alpha})+p_1^*\mathcal{O}_{J}(\theta_{\alpha})+p_2^*\mathcal{O}_{J}(\theta_{\alpha})$$
where $m:J \times J\longrightarrow J$ is the multiplication in the group law of $J$.\footnote{We replace the notation $\oplus$ by the addition for the convenience of both reading and typing. We will also do this in the following lemma.}
\begin{center}
		\begin{tikzcd}
		J \arrow[rd, "{(\Id,\beta)}"] &              & J \\
		J \arrow[r, "\Delta"]     & J\times J \arrow[ru, "p_1"] \arrow[rd, "p_2"'] \arrow[r, "m"]& J  \\
		J \arrow[ru, "{(\beta,\Id)}"'] &              & J
		\end{tikzcd}
\end{center}
\begin{lemma}
Consider the canonical line bundles as follows:
\begin{center}
% https://tikzcd.yichuanshen.de/#N4Igdg9gJgpgziAXAbVABwnAlgFyxMJZABgBoBGAXVJADcBDAGwFcYkQAdDgW3pwAsAxk2AB5AL4B9AMIAKAB4BKEONLpMufIRTlSAZmp0mrdtJVqQGbHgJEATBUMMWbRJw4RuMAOb0Z59WstIj1SYidjV3deAWFGMSkAKVlkAFoqAD0AKgACLgEYHD9gLiY0fnpxZVVAzVsUABZSOwiXdkSAyw0bbWQAVjDWkzcuGKERCUlk-P5C4tLGcsrqiys63oA2RxpnYei+cfjJ4GkMu3FZLgARGEYildqeoi2DHcjTPI48LzgcsxqukF6sgAJyDN5tNwABU6ayeKDBLQhe0Sn2+8ByHXEhhgUG88CIoAAZgAnTxIMggHAQJC6ECzehQdiQMBsAGk8mIBxUmmIOkMpluFlsiwc7hIUI8pBNekwRnMggi4lk8WIAZSxAygUK1mdMVILYagDsNG1QsVepVSAAHDRqUgwbL5ebdeyrZq7byQW7OSaNY7dlFZFhJCUOGUKqoQ2GI8tLZy6faPUZISBowsluJ46rE7y-WawMxGIxseIgA
\begin{tikzcd}[column sep={10mm,between origins}]
  &  [-3mm]      & [-3mm]   &[3mm] {\mathcal{O}_J([-1]^* \theta_{\alpha})} \arrow[rdd, no head] &                 & \mathcal{O}_J(\theta_{\alpha}) \arrow[ldd, no head] &                  &  &  & P \arrow[dd, no head] \\
\mathcal{O}_C(x) \arrow[rdd, no head] &        & \omega_C \arrow[ldd, no head] &&                 &                 & \mathcal{O}_{C^2}(\Delta)         &  &  &                       \\[-5mm]
  &        &    && J \arrow[rrrrr] &                 &                  &  &  & J \times J            \\
  & C \arrow[rrru, "i_{\alpha}"'] \arrow[rrrrr] &    &&                 &                 & C \times C \arrow[rrru, "{(i_{\alpha},i_{\alpha})}"']\arrow[from=uu, no head, crossing over] &  &  &                      
\end{tikzcd}
\end{center}
We can write down their pullback explicitly:
\begin{equation}
\begin{aligned}
  i_{\alpha}^* \Big(\mathcal{O}_J(\theta_{\alpha}) \Big)\cong\;& \omega_C+(2-g)\alpha  \\ 
  i_{\alpha}^* \Big(\mathcal{O}_J([-1]^* \theta_{\alpha}) \Big)\cong\;& g\alpha \\   
  \Delta^* \Big(\mathcal{O}_{C^2}(\Delta) \Big)\cong\;& -\omega_C \\ 
  (x,\Id)^* \Big(\mathcal{O}_{C^2}(\Delta) \Big)\cong\;& \mathcal{O}_C(x) \\  
  (i_{\alpha},i_{\alpha})^* \big(P\big)\cong\;&  \mathcal{O}_{C^2}(\Delta)-p_1^*\alpha-p_2^*\alpha\\  
  \Delta_J^* \big(P\big)\cong\;&-\mathcal{O}_J(\theta_{\alpha})- \mathcal{O}_J([-1]^* \theta_{\alpha}) \\ 
  (\beta,\Id)^* \big(P\big)\cong\;& -T_{\beta}^*\mathcal{O}_J(\theta_{\alpha})+\mathcal{O}_J(\theta_{\alpha}) \\ 
\end{aligned}
\end{equation}
\end{lemma}
\begin{proof}[Proof?]
That's a good exercise for Abelian variety. See \cite[Lemma A.3]{yuan2021arithmetic}.\footnote{If you really check the details, you'll find he refers to Serre's book, and in Serre's book he refers to the Weil's book which is in French and can't be found in the internet. Surprise! I will write down the proof when I fully understand it.}
\end{proof}

\bibliographystyle{plain}
\bibliography{reference}
\end{document}




